\documentclass[a4paper, fontsize=14pt]{article}
\usepackage{scrextend}
\usepackage{indentfirst, fancyhdr, amsfonts, mathtools, amssymb}
\usepackage{titlesec} %работа с рубрикацией
\usepackage{tocloft} %настройки оглавления
\usepackage[T2A]{fontenc}
\usepackage[utf8x]{inputenc}
\usepackage[russian]{babel}
\usepackage{hyperref} %кликабельное оглавление
\usepackage[left=3.7cm,right=2cm,top=2cm,bottom=2cm]{geometry}
\usepackage{tempora} %настраиваем шрифт типа TNR                                   
\usepackage{newtxmath} %делаем шрифт формул похожим на TNR
\usepackage{caption}
\usepackage{pdfpages}
\usepackage{listings}
\lstset{
  columns=fullflexible,
  breaklines=true,
}
\linespread{1}
\setcounter{page}{4} %в зависимости от того, какой по счёту страницей должно быть оглавление!

%НАСТРОЙКИ ОГЛАВЛЕНИЯ
\renewcommand{\cftsecaftersnum}{.} %точки после номеров разделов и подразделов в оглавлении
\renewcommand{\cftsubsecaftersnum}{.}
\renewcommand{\cftsecfont}{\normalfont} %разделы в оглавлении пишутся обычным (не жирным) шрифтом
\renewcommand{\cftsecpagefont}{\normalfont} %соответствующие им страницы тоже
\renewcommand{\cftsecleader}{\cftdotfill{\cftdotsep}} %расставляем точки между названиями разделов и их страницами
\addto\captionsrussian{\renewcommand\contentsname{СОДЕРЖАНИЕ}} %хотим, чтобы слово "Содержание" писалось капсом
\renewcommand{\cfttoctitlefont}{\hfil\bfseries} %слово СОДЕРЖАНИЕ по центру жирным
\renewcommand{\cftaftertoctitle}{\hfill}

%НАСТРОЙКИ РУБРИКАЦИИ
\titleformat*{\section}{\center\bf} %названия разделов и подразделов по середине жирным шрифтом
\titleformat*{\subsection}{\center\bf}
\titlelabel{\thetitle.\quad} %название раздела и его номер отделены точкой

%НАСТРОЙКИ БИБЛИОГРАФИИ
\addto\captionsrussian{\renewcommand\refname{СПИСОК ЛИТЕРАТУРЫ}} %хотим, чтобы слова "Список литературы" писались капсом
\makeatletter
\renewcommand{\@biblabel}[1]{#1.} %хотим, чтобы в списке литературы номера источников писались в формате "No. <...>", а не "[No] <...>"
\makeatother

\begin{document}
% \includepdf[pages={1}]{src/front_page.pdf}
\textbf{Цель работы:}  получить навык проведения вычислительного
эксперимента, направленного на исследование свойств итерационных методов
решения СЛАУ.
\subsection*{{Ход работы}}
   \newpage
\section*{{Вывод}}
Вьходе лабораторной работы был получен навык проведения вычислительного
эксперимента, направленного на исследование свойств итерационных методов
решения СЛАУ.
\newpage
\subsection*{Список литературы}
\begin{enumerate}
    \item Бахвалов Н.С., Жидков Н.П., Кобельков Г.М. Численные методы: Бином, 2018. – 636 с. 
    \item Калиткин Н.Н. Численные методы, 2-е издание: БХВ-Петербург, 2014. – 592 с.
    \item Самарский А.А., Гулин А. В. Численные методы: Учеб, пособие для вузов, — М.: Наука. Гл. ред. физ-мат. лит., 1989.— 432 с.
\end{enumerate}
\newpage
\subsection*{Приложение}
Весь С++ код выложен в github-репозитории по ссылке: 

% \url{https://github.com/sultanovMF/Numerical-Methods-Lab}


\end{document}